\documentclass[a4paper,10pt]{article}
\usepackage[utf8]{inputenc}
\usepackage{minted}

\begin{document}

\section*{Aufgabe 1: Verständnisfragen}
\begin{enumerate}
 \item Im Kernmodus wird mit unbeschränkten Berechtigungen ausgeführt: Man kann jede Funktion aufrufen, auf jede Ressource zugreifen und nutzt Kernthreads. Im Benutzermodus hat man nur beschränkte Rechte und alle Aktionen die solche erfordern werden über den Kernmodus via System Call Interface ausgeführt.
 \item Welche Befehle sollten nur im Kernmodus ausgeführt werden?
 \begin{itemize}
  \item Nur Kernmodus, da der Hardware zugriff dadurch geregelt wird
  \item Auch Benutzermodus, die Systemzeit zu lesen beeinflusst keinen anderen Prozess oder das Betriebssystem
  \item Nur Kernmodus, da die Systemzeit viele andere Prozesse beeinflusst
  \item Nur Kernmodus, da hier Code und Daten anderer Prozesse direkt beeinflusst werden
 \end{itemize}
 \item Vorteile: Konsekutive logische Adressierung, möglich mehr Arbeitsspeicher zu verfügung zu stellen als vorhanden, mehrere Programme können gleichzeitig ausführen mit geringem Mehraufwand. Nachteile: Mehraufwand beim Laden, komplexere Speicherverwaltung, z.T. kein bulk laden möglich
 \item Lesen von der Datei bevor zuende geschrieben wurde führt zu einem invaliden read. Der Treiber könnte dieses Problem durch locks lösen bspw. durch Read-Write Locks.
\end{enumerate}

\section*{Aufgabe 2: Selbststudium}
\begin{enumerate}
 \item Charaktersitisch für Betriebssystem-Art:
    \begin{itemize}
     \item Mehrere veschiedene Anwendung mit Nutzeroberfläche müssen gleichzeigt ausgeführt werden \& die Nutzeroberfläche muss immer ``Responsive'' sein, d.h. in kurzmöglichster Zeit reagieren. Meist nur ein oder wenige Nutzer die physisch vor dem Gerät sitzen. Häufige Änderung der Konfiguration (wie Programme installieren, Downloads,...)
     \item Oft real-zeit kritische Anwendungen meist ohne Nutzeroberfläche müssen sehr robust und zuverlässig Funktionieren. Wird oft in Maschienen eingesetzt, wie z.B. Autos oder Robotern. Sollte nach einmaligem Setup nicht mehr regelmässig geändert werden müssen.
     \item Sendet Daten über das Netzwerk an einen Empfänger oder speichert große Mengen der gemeßenen Werte. Führt meist nur einen Prozess aus pro Sensor und keine weiteren Funktionen. Zugriff haben meist nur wenige Zugriffsberechtigte.
     \item Wird oft in Kombination mit eingebetteten Systemen genutzt. Immer dann wenn es essenziell ist, dass das System reagiert wie z.B. beim Bremsen eines Autos
    \end{itemize}
    \item Ein Time-Sharing System führt die Prozesse eines Nutzers eine festgelegte Zeit aus und bedient dann den nächsten Nutzer. Ein Multiprogrammsystem führt einen Prozess eine festgelegte Zeit aus und bedient dann den nächsten Prozess undabhängig vom Nutzer.
\end{enumerate}

\section*{Aufgabe 3: Einheiten}
\begin{enumerate}
 \item 
 \begin{tabular}{c|c|c|c}
 yobi & Yi & 80 & 8 \\
 zebi & Zi & 70 & 7 \\
 exbi & Ei & 60 & 6 \\
 pebi & Pi & 50 & 5 \\
 tebi & Ti & 40 & 4 \\
 gibi & Gi & 30 & 3 \\
 mebi & Mi & 20 & 2 \\
 kibi & Ki & 10 & 1 \\
 \end{tabular}
\item Weil halbe Bits physisch nicht repräsentierbar sind. Komma Zahlen werden anders codiert. Üblicherweise Größenangaben für Speicher.
\item 0,909495
\end{enumerate}

\section*{Aufgabe 4: Ausführungs- und Zugriffszeiten}
\begin{enumerate}
 \item 
 \begin{itemize}
  \item 333333 Befehle pro Sekunde
  \item 200000 Befehle pro Sekunde
 \end{itemize}
 
 \item 
 \begin{enumerate}
  \item $0.95 \cdot 2ns + 0.05 \cdot (0.99 \cdot 10ns + 0.01 \cdot 10000000ns) = 5002.4 ns$
  \item Die Zugriffszeit zum externen Speicher, da sie um das 5000000-fache größer ist als die Cache Zugriffszeit bzw. das 1000000-fache als die Arbeitsspeicherzugriffszeit. Ausserdem sind 95\% Cache-Hit rate sehr optimistisch.
  \item $0.95 \cdot 2ns + 0.05 \cdot (0.99 \cdot 10ns + 0.01 \cdot 4 ms) = 2002,395 ns$
 \end{enumerate}
\end{enumerate}

\section*{Aufgabe 5: Hello World!}
\begin{enumerate}
 \item Siehe hello.c
 \item 
 \begin{itemize}
  \item \mintinline{bash}{-std=c11} setzt den C standard bzw. die verfügbaren Sprachkonstrukte auf den Iso standard von 2011
  \item \mintinline{bash}{-g} generiert Debugging Informationen im nativen Format des Betriebsystems z.B. DWARF für Unixoide
  \item \mintinline{bash}{-Wall} aktiviert die meisten Warnungen des Compilers
  \item \mintinline{bash}{pedantic} aktiviert weitere Warnungen des Compilers
 \end{itemize}
    \item .bc --- Bitstream file, containing an abstract encoding of the intermediate representation in binary form
    \item .i --- a preprocessed file: Macros are expanded and preprocessor direcitves are executed.
    \item .s --- An assembly file, architecture specific instructions
    \item .o --- A machine code file, binary format, specific to the architecture, generated by execution of the assembler with the .s file as input
    \item .out --- linked machine code file, ready for execution, gained by execution of the linker with the .o file as input
 
\end{enumerate}


\section*{Aufgabe 6}
\begin{enumerate}
 \item It tells the linker to link against the c math library
 \item store the image in a two dimensional array and print it.
\end{enumerate}

\end{document}
